\documentclass[letterpaper]{article}

\usepackage{natbib,alifeconf}  %% The order is important
\usepackage{amsfonts}

\newcommand{\gv}{$g$-vector}
\newcommand{\phv}{$\phi$-vector}

\title{Acclivation of Virtual Fitness Landscapes}
\author{Ben Kovitz$^{1}$, Dave Bender$^{1}$, \and Marcela Poffald \\
\mbox{}\\
$^1$Fluid Analogies Research Group, Indiana University, Bloomington, IN 47408 \\
bkovitz@indiana.edu}

\begin{document}
\maketitle

\begin{abstract}
Simple spreading-activation networks are shown to exhibit acclivation of a
``virtual fitness landscape''. That is, the fitness function faced by part of
the genome, when the remainder of the genome is held constant, becomes
progressively more smoothly sloped, or hill-shaped.

Applied directly to the phenotype, the original fitness function is filled
with traps and local optima. The topology of the spreading-activation network
experiences indirect selective pressure to make the fitness landscape become
less rugged as seen by a numerical vector that is fed into the network.
The evolving topology exploits regularities in the original fitness
function, such as a ``ridge'' and regularly spaced ``bumps'', to make it
easier for the vector to navigate the fitness landscape by simple
hill-climbing.

This shows that under appropriate conditions, complex \mbox{epistasis} and
non-locality of genotype--phenotype mapping can improve evolvability. These
conditions are typical of those found in nature, where behavior and phenotypes
are produced mainly by cascades of activity through a graph and coordination
among the parts of the phenotype is a precondition of successful survival and
reproduction.

\end{abstract}

% Relevant topics to search in the literature:
%
% how much epistasis is there in natural genomes?
% crosstalk in metabolic networks

\section{Virtual Fitness Functions}
@@.\footnote{\cite{hogeweg2012toward}.}

Any part of a genome is selected against a \textit{virtual fitness function}
resulting from the rest of the genome and the fitness function faced by the
genome as a whole. If the whole-genome fitness function reflects the influence
of the environment on the genome, then the virtual fitness function represents
the same for a part of the genome, whose environment includes the rest of
the genome.

Let a set of genotypes $G$ have a mapping $m_{G} : G \rightarrow \Phi$ to a set
of phenotypes $\Phi$, and let $w_\Phi : \Phi \rightarrow \mathbb{R}$ be the
fitness function for the phenotypes. Then $w_G : G \rightarrow \mathbb{R}$, the
fitness function for the whole genome, is the composition of these functions,
$w_\Phi(m_{G}(g))$ for $g$ in $G$.

If we divide the genome into two parts, then each genotype $g \in G$ consists
of a $g_1 \in G_1$ and a $g_2 \in G_2$, in which each $g_2$ defines a
partial-genotype--phenotype mapping $m_{g_2} : G_1 \rightarrow \Phi$. That is,
if we hold part of the genome constant, say by fixing $g_2$, this defines a
mapping from all possible values of the rest of the genome, $g_1$, to
corresponding phenotypes. If we reverse $g_1$ and $g_2$, then of course we get
the opposite partial-genotype--phenotype mapping, $m_{g_1} : G_2 \rightarrow
\Phi$.

These mappings, in turn, define virtual fitness functions $v_{g_2}(g_1) =
w_\Phi(m_{g_2}(g_1))$ and $v_{g_1}(g_2) = w_\Phi(m_{g_1}(g_2))$. As mutations
and crossovers can alter either or both of $g_1$ and $g_2$, the partial
genomes $G_1$ and $G_2$ coevolve cooperatively, each selected by
the fitness functions $v_{g_1}$ and $v_{g_2}$, which vary among all the
individuals and vary each generation.

Let \textit{evolvability} be defined in some reasonable way (there are many),
so that greater evolvability implies some advantage in navigating a fitness
landscape upward faster or further over succeeding generations. Let $g_a,g_b
\in G$ be two individuals in the same population and the same generation.
Assuming no other advantages favoring either $g_{a1}$ or $g_{b1}$, if $g_{a1}$
presents its mate $g_{a2}$ with a virtual fitness function $v_{g_{a1}}$ that
$g_{a2}$ finds more evolvable than $v_{g_{b1}}$ is for $g_{b2}$, then the
descendants of $g_a$ will evolve faster or further than the descendants of
$g_b$ (according to how evolvability is defined).

Therefore each partial genome is under selective pressure to create a virtual
fitness landscape for the other partial genome that gives the latter greater
evolvability. To illustrate with an unrealistically simple example, if the eye
and the arm are governed by separate sets of genes, and some arm shapes make
it easier for the eye to evolve, then there is selective pressure favoring
alleles for those arm shapes. All other factors being equal, evolution favors
arms that make eyes easier to evolve. This selective pressure happens
indirectly; in one generation, greater fitness wins. But over successive
generations, descendants of organisms with greater evolvability will tend to
have greater fitness than organisms with lesser evolvability.@@cite-Altenberg?

The above considerations make no difference for homogeneous genomes, where
every part of each genotype undergoes mutation and crossover the same as every
other part and exerts the same effect on the phenotype or on the total fitness
as every other part. However, if $G_1$ and $G_2$ vary according to different
operators and/or affect the phenotype or total fitness differently, there is
potential for each part to seek values that make the other part more
evolvable, resulting in a period of progressively increasing evolvability for
the organism as a whole.

In the rest of this paper, we examine a natural and common way for this
synergy to occur: when $g_1$ consists of a vector of real numbers (``knobs'')
and $g_2$ consists of a network that provides connections through which the
numbers from $g_1$ interact.

\section{Acclivation}

As is well known, a genome consisting of a vector of numbers, where mutations
alter the numbers by small amounts, evolves most easily against a hill-shaped
fitness function. In a hill-shaped fitness function, local increases in
fitness correlate perfectly with movement toward the peak of the whole fitness
landscape. The more ``rugged'' the landscape, the weaker is this correlation,
so that following the local gradient can lead organisms to become stuck at
local maxima, which might be very low, and from which they cannot escape by
local mutations (though they might escape by crossover).@@cite-Kauffman

Therefore, if a vector of numbers faces a rugged fitness landscape, with
difficult features such as low local peaks and impassable moats, we can
improve its evolvability for a vector of numbers by making its fitness
function more hill-shaped. Let us call the process of making a fitness
landscape more hill-shaped \textit{acclivation.}

So, in a genome where $G_1$ is a vector of numbers that mutate by small
amounts, and $G_2$ is a directed graph that feeds the numbers in $G_1$ through
nodes that perform some function on the numbers from their input edges,
eventually leading to a phenotype whose fitness determines the fate of the
whole organism, we should expect selective pressure for genotypes $g_2 \in
G_2$ to produce mappings that induce acclivation on the virtual fitness
functions $v_{g_2}$. Evolution should favor graphs that put knobs in a
position where they can hill-climb successfully.

\subsection{Genome for Experimentation}

To test the preceding hypothesis in a form in which acclivation will be
visually apparent on plots printed on paper, we limit ourselves to genomes
where $g_1$ and the phenotype are 2-dimensional vectors and $g_2$ is a graph
connecting them.  The whole genome is a directed graph where:
\begin{enumerate}
   \item Two nodes, called the \textit{knobs}, $k_1$ and $k_2$, are designated
      to each hold a number in $[-1.0, +1.0]$, called an \textit{initial
      activation}.
   \item Two other nodes, $p_1$ and $p_2$, are designated to hold the
      phenotype.
   \item Each edge has a weight of either $+1.0$ or $-1.0$.
\end{enumerate}

The phenotype is determined by a process of spreading activation, run for
10 timesteps. Each timestep, the initial activations from the knobs flow into
adjacent nodes, producing a number in $[-1.0, +1.0]$ called the node's
\textit{activation}. The activation of a node $a_j$ at timestep $t+1$ is
calculated according to the following function:
\[
   a_j(t+1) = T(a_j(t) + \sum_iW_{ij}a_i(t))
\]
where $W_{ij}$ is the weight of the incoming edge, if any, from node $i$ to
node $j$, and $T$ is the following transfer function:
\[
   T(x) = \frac{2}{1+\exp(-Sx)}-1
\]
where $S=2.1972274554893376$. This constant gives $T$, when iterated,
attractors at $\pm0.5$ and a repellor at 0.

If a node does not have an activation at time~$t$, then it does not figure
into the above sum for calculating any other node's activation. At time~$t=0$,
only the knobs have activations.

The phenotype is the vector $(a_{p_1}(10), a_{p_2}(10))$, i.e.~the activations
of the phenotype nodes after 10 timesteps. If $p_1$ or $p_2$ has not received
an activation after 10 timesteps, then the genotype has no phenotype and is
given a fitness of~0.0. This can happen if no edges provide a path from a knob
to $p_1$ or~$p_2$.

@@Explain mutation and crossover

\section{Experiments}

In each experiment, we run the genome defined above, perhaps with a slight
variation, against a different family of fitness functions, and see what
virtual fitness functions emerge. We only plot $v_{g_2}$, i.e.~the virtual
fitness function seen by the knobs, since we know of no way to plot fitness
functions seen by a graph.

We must say ``family'' of fitness functions, because each experiment's fitness
function has a constant that changes randomly once per epoch, i.e.~every 20
generations. This constant moves the peak of the fitness function to different
places in phenotype space.

Frequently moving the peak is necessary to induce strong selective pressure
favoring evolvability rather than organisms that only approximate the current
peak.@@cite Each new epoch is thus a race to approach the new peak. The most
evolvable genotypes' lineages will approach the new peak faster than their
less evolvable competitors. In this way, the experiments tend to reward
evolvability especially strongly.

\subsection{Experiment 1: The Narrow Ridge}

In this experiment, we run the experimental genome against this fitness
function, plotted in @@fig:
\[
   w(\phi) = 
\]


\subsection{Experiment 2: The Razorback}

\subsection{Experiment 3: The Circle}

\subsection{Experiment 4: Moats}

\section{Discussion}

@@genetic memory

\subsection{Significance of the Transfer Function}

\section{Conclusions}


\bibliographystyle{apalike}
\bibliography{acclivation} % replace by the name of your .bib file

\end{document}
