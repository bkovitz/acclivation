\documentclass[letterpaper]{article}

\usepackage{natbib,alifeconf}  %% The order is important

\newcommand{\gv}{$g$-vector}
\newcommand{\phv}{$\phi$-vector}

\title{Acclivation of a Virtual Fitness Landscape}
\author{Ben Kovitz$^{1}$, Dave Bender$^{1}$, \and Marcela Poffald \\
\mbox{}\\
$^1$Fluid Analogies Research Group, Indiana University, Bloomington, IN 47408 \\
bkovitz@indiana.edu}

\begin{document}
\maketitle

\begin{abstract}
Simple spreading-activation networks are shown to exhibit acclivation of a
``virtual fitness landscape''. That is, the fitness function faced by part of
the genome, when the remainder of the genome is held constant, becomes
progressively more smoothly sloped, or hill-shaped.

Applied directly to the phenotype, the original fitness function is filled
with traps and local optima. The topology of the spreading-activation network
experiences indirect selective pressure to make the fitness landscape become
less rugged as seen by a numerical vector that is fed into the network.
The evolving topology exploits regularities in the original fitness
function, such as a ``ridge'' and regularly spaced ``bumps'', to make it
easier for the vector to navigate the fitness landscape by simple
hill-climbing.

This shows that under appropriate conditions, complex \mbox{epistasis} and
non-locality of genotype--phenotype mapping can improve evolvability. These
conditions are typical of those found in nature, where behavior and phenotypes
are produced mainly by cascades of activity through a graph and coordination
among the parts of the phenotype is a precondition of successful survival and
reproduction.

\end{abstract}

% Relevant topics to search in the literature:
%
% how much epistasis is there in natural genomes?
% crosstalk in metabolic networks

\section{Introduction}
Stuff.\footnote{\cite{hogeweg2012toward}.}

\section{Experiments}

Each genotype is a directed graph containing at least four nodes, designated
$g_1, g_2, p_1,$ and $p_2$. Additional nodes are designated $n_1, n_2, n_3,
\ldots$. Associated with each node other than $p_1$ and $p_2$ is an ``initial
activation'': a number in the range $[-1.0,+1.0]$.

The genotype--phenotype mapping consists of running the following
spreading-activation algorithm on the graph, and considering the resulting
acclivations of $p_1$ and $p_2$ the phenotype.

The initial activations of $g_1$ and $g_2$ are called the \gv, i.e. the
genotype vector. The activations of $p_1$ and $p_2$ are called the
\phv, i.e. the phenotype vector.

@@fitness function $w$

\section{Conclusions}



\bibliographystyle{apalike}
\bibliography{acclivation} % replace by the name of your .bib file

\end{document}
